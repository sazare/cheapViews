
\subsection{用語あるいは定義}
\begin{description}
\item[観測命題:] 世界を観測して得られた事実とみなせる論理式。ここでは観測と呼んでいるが、数理論理学では公理と呼ばれているもの。(ここ、意味論と証明論がまざっている)公理というほど確定したものではなく、観測し正しいと信じられる言明程度のもの。
\item[観測集合:] 観測命題の集合。記号では$\mathcal{A}$で表すと思う。
\item[仮説:] 証明しようとする論理式。conjectureと呼んだり、conjと書いたりする。記号では$\Psi(x)$のように書くことが多いはず。
\item[Fact:] 観測論理式の中で、観測された事実のみの論理式やその集合を意味する。真実という意味ではない。ground unit clause になる。
\item[定義:] 観測集合に書かれた、Factの間の関係を意味する論理式。indubtive definition も含むが、定義という役割にすることで、真偽を証明する義務を除外したものが定義。
\item[エルブラン宇宙] エルブランドメインと呼ぶかも。特定の述語記号に対するエルブランドメインについても書くかもしれない。
\end{description}


観測命題はポジティブのみである。
なぜなら、否定された状態というものは、直接、観測できず、何らかの推論によってのみ得られる状態だから。


\subsection{定義}

$\cont$は 矛盾を示す。

$\mathcal{H_{\mathcal{A}}}$は、観測集合$\mathcal{A}$に対するエルブラン宇宙

$\clos{A} := \{x | \mathcal{A} \vdash x\}$




